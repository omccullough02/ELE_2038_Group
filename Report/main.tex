\documentclass[a4paper,10pt,reqno]{amsart}

\usepackage[utf8]{inputenc}
\usepackage[foot]{amsaddr}
\usepackage{amsmath,amsfonts,amssymb,amsthm,mathrsfs,bm}
\usepackage[margin=0.95in]{geometry}
\usepackage{color}
\usepackage[dvipsnames]{xcolor}

\input{toc-config.tex}

\usepackage{mathtools,enumerate,mathrsfs,graphicx}
\usepackage{epstopdf}
\usepackage{hyperref}

\usepackage{latexsym}


\definecolor{CommentGreen}{rgb}{0.0,0.4,0.0}
\definecolor{Background}{rgb}{0.9,1.0,0.85}
\definecolor{lrow}{rgb}{0.914,0.918,0.922}
\definecolor{drow}{rgb}{0.725,0.745,0.769}

\usepackage{listings}
\usepackage{textcomp}
\lstloadlanguages{Matlab}%
\lstset{
    language=Matlab,
    upquote=true, frame=single,
    basicstyle=\small\ttfamily,
    backgroundcolor=\color{Background},
    keywordstyle=[1]\color{blue}\bfseries,
    keywordstyle=[2]\color{purple},
    keywordstyle=[3]\color{black}\bfseries,
    identifierstyle=,
    commentstyle=\usefont{T1}{pcr}{m}{sl}\color{CommentGreen}\small,
    stringstyle=\color{purple},
    showstringspaces=false, tabsize=5,
    morekeywords={properties,methods,classdef},
    morekeywords=[2]{handle},
    morecomment=[l][\color{blue}]{...},
    numbers=none, firstnumber=1,
    numberstyle=\tiny\color{blue},
    stepnumber=1, xleftmargin=10pt, xrightmargin=10pt
}

\numberwithin{equation}{section}
\synctex=1

\hypersetup{
    unicode=false, pdftoolbar=true, 
    pdfmenubar=true, pdffitwindow=false, pdfstartview={FitH}, 
    pdftitle={ELE2024 Coursework}, pdfauthor={A. Author},
    pdfsubject={ELE2024 coursework}, pdfcreator={A. Author},
    pdfproducer={ELE2024}, pdfnewwindow=true,
    colorlinks=true, linkcolor=red,
    citecolor=blue, filecolor=magenta, urlcolor=cyan
}

% CUSTOM COMMANDS
\renewcommand{\Re}{\mathbf{re}}
\renewcommand{\Im}{\mathbf{im}}
\newcommand{\R}{\mathbb{R}}
\newcommand{\N}{\mathbb{N}}
\newcommand{\C}{\mathbb{C}}
\newcommand{\lap}{\mathscr{L}}
\newcommand{\dd}{\mathrm{d}}
\newcommand{\smallmat}[1]{\left[ \begin{smallmatrix}#1 \end{smallmatrix} \right]}

%opening
\title[ELE2038 Coursework]{ELE2038: Group Coursework Report}

\author[O. McCullough]{Owen McCullough}
\author[O. Quail]{Orla Quail}
\author[E. Acton]{Eoin Acton}

\begin{document}

\maketitle

\section*{Introduction}
In this report we aim to develop a controller for a wooden ball on an incline plane. The system can be seen below:
\begin{figure}[h]
\centering
\includegraphics[width=0.6\linewidth]{x_forced_response.eps}
\caption{Ball on an Incline Plane}
\label{fig:system}
\end{figure}

\section{Derivation of System Equations}

\subsection{Modelling}
From \hyperref[fig:system]{Figure 1}, the force $F$ acting on the ball can be written as:
\begin{equation}
\label{sec:eq11}
    F_{total}=F_{magnet}+F-F_{spring}-F_{dampener}
\end{equation}
The force of gravity acting on the ball can be written as:
\begin{equation*}
    F=mg
\end{equation*}
where $m$ is the mass of the ball, and $g$ is the gravitational constant. Due to the inline plane, we have to split this force into horizontal and vertical components:
\begin{equation*}
    F=mg\cos{\phi}+mg\sin{\phi}
\end{equation*}
We can see in \hyperref[fig:system]{Figure 1} that $mg\cos{\phi}$ will be equal to 0 as it cannot move vertically downwards, therefore:
\begin{equation*}
    F=mg\sin{\phi}
\end{equation*}
Combining $F$ with the secondary force $T$, we can begin solving for an equation of motion:
\begin{equation}
\label{sec:eq12}
    mg\sin{\phi}-T=ma=m\ddot x
\end{equation}
The equation for the force $T$, can be written as:
\begin{equation*}
    T=\frac{M}{r}
\end{equation*}
Since the torque $M$ is defined as the moment of inertia $I$ multiplied by the angular acceleration $\ddot \theta$, we can write:
\begin{equation}
\label{sec:eq13}
    T=\frac{I\ddot \theta}{r}
\end{equation}
Combining the above equation for $T$ (\ref{sec:eq13}) with (\ref{sec:eq12}), we get:
\begin{gather*}
    \notag mg\sin{\phi}-\frac{I\ddot \theta}{r}=ma=m\ddot x, \\
    \notag mg\sin{\phi}=(\frac{I}{R}+mR)\ddot \theta, \\
\end{gather*}  
Substituting the equation for moment of inertial for a sphere, we get:
\begin{align*}
    \notag mg\sin{\phi}&=(\frac{2}{5}+1)mR\ddot \theta, \\
    \notag mg\sin{\phi}&=\frac{7}{5}m\ddot x, \\
\end{align*}
Therefore:
\begin{equation}
\label{sec:eq14}
        F=g\sin{\phi}-\frac{7}{5}\ddot x
\end{equation}
For the spring and dampener forces, they can be modelled as the following equations:
\begin{subequations}
\label{sec:eq15}
\begin{align}
    F_{spring}&=k(x-d), \\
    F_{dampener}&=b\dot x
\end{align}
\end{subequations}
Finally, in the problem statement we were given, $F_{magnet}$ was defined as:
\begin{equation}
\label{sec:eq16}
    F_{magnet}=c^2\frac{i^2}{y^2}
\end{equation}
where $y=(\delta-x)$. 
\subsection{Obtaining State Space Representation} Combining our force equations (\ref{sec:eq14}, \ref{sec:eq15} and \ref{sec:eq16}) we get:
\begin{equation}
\label{sec:eq17}
    \frac{7}{5}m\ddot x=c\frac{i^2}{(\delta-x)^2}+mg\sin{\phi}-k(x-d)-b\dot x
\end{equation}
To obtain SSR, we create 3 new variables:
\begin{align*}
x_1&=x, \\
x_2&=\dot x, \\
x_3&=i
\end{align*}
Substituting these into (\ref{sec:eq17}) gives us the following SSR:
\begin{equation}
\label{sec:eq18}
    \bold{\dot x} = f(\bold x, V) = \begin{bmatrix} 
        \dot x_1 \\ 
        \dot x_2 \\ 
        \dot x_3 
    \end{bmatrix} = \begin{bmatrix}
        x_2 \\
        \frac{5}{7m}(c\frac{{x_3}^2}{(\delta-x_1)^2}+mg\sin{\phi}-k(x_1-d)-bx_2) \\
        \frac{V-x_3R}{L_0+L_1\exp{-\alpha(-\delta-x_1)}}
    \end{bmatrix}
\end{equation}

\section{Linearisation of System Equations}
The definitions for the dynamical system derived in (\ref{sec:eq18}) can be written as:
\begin{align}
\notag \dot x_1 &= x_2,                  \\
\notag \dot x_2 &= \theta(x_1, x_2, x_3),  \\
\notag \dot x_3 &= \psi(V, x_1, x_3)
\end{align}
Where the functions $\theta(x_1, x_2, x_3)$ and $\psi(V, x_1, x_3)$ represent the acceleration of the ball and the rate of change of the current in the electromagnet respectively.
To obtain a linear approximation for $\theta(x_1, x_2, x_3)$ and $\psi(V, x_1, x_3)$ around the equilibrium points $(x_1^e, x_2^e, x_3^e, V^e)$, we derived the following partial derivatives for both functions:
\begin{subequations}
\label{sec:eq21}
\begin{align}
    A = \frac{\partial \theta(x_1,x_2,x_3)}{\partial x_1}\bigg|_{x_1^e, x_2^e, x_3^e} &= \frac{5 \cdot \left(\frac{2 c {x_{3}^e}^{2}}{\left(\delta - x_{1}^e\right)^{3}} - k\right)}{7 m}, \\
    B = \frac{\partial \theta(x_1,x_2,x_3)}{\partial x_2}\bigg|_{x_1^e, x_2^e, x_3^e} &= - \frac{5 b}{7 m}, \\
    C = \frac{\partial \theta(x_1,x_2,x_3)}{\partial x_3}\bigg|_{x_1^e, x_2^e, x_3^e} &= \frac{10 c x_3^e}{7 m \left(\delta - x_1^e\right)^{2}}
\end{align}
and
\begin{align}
    D = \frac{\partial \psi(V,x_1,x_3)}{\partial x_1}\bigg|_{V^e, x_1^e, x_3^e} &= 0, \\
    E = \frac{\partial \psi(V,x_1,x_3)}{\partial x_3}\bigg|_{V^e, x_1^e, x_3^e} &=- \frac{R}{L_{0} + L_{1} e^{- \alpha \left(\delta - x_{1}^e\right)}}, \\
    F = \frac{\partial \psi(V,x_1,x_3)}{\partial V}\bigg|_{V^e, x_1^e, x_3^e} &=\frac{1}{L_{0} + L_{1} e^{- \alpha \left(\delta - x_{1}^e\right)}}
\end{align}
\end{subequations}
Taylor's theorem then gives us the following approximations for $\theta(x_1, x_2, x_3)$:
\begin{equation}
\label{sec:eq22}
    \theta \approx \theta^{e} + A(x_1-x_1^e) + B(x_2-x_2^e) + C(x_3-x_3^e)
\end{equation}
and for $\psi(V, x_1, x_3)$:
\begin{equation}
\label{sec:eq23}
    \psi \approx \psi^{e} + D(x_1-x_1^e) + E(x_3-x_3^e) + F(V-V^e)
\end{equation}
We then introduced deviation variables, which are defined as $\bar x=(x-x^e)$, and rewrote equations (\ref{sec:eq22}) and (\ref{sec:eq23}) substituting in deviation variables for $x_1$, $x_2$, $x_3$ and $V$:
\begin{align}
\label{sec:eq24}
    \theta &\approx \theta^e + A\bar x_1 + B\bar x_2 + C\bar x_3, \\
    \psi &\approx \psi^e + D\bar x_1 + E\bar x_3 + F\bar V
\end{align}
This allows us to rewrite the SSR (\ref{sec:eq18}) we obtained in the previous section in the following linear form:
\begin{equation}
\label{sec:eq26}
    \bold{\dot \bar{x}} = f(\bold x, V) = \begin{bmatrix} 
        \dot x_1 \\ 
        \dot x_2 \\ 
        \dot x_3 
    \end{bmatrix} \approx \begin{bmatrix}
        x_2 \\
        A\bar x_1 + B\bar x_2 + C\bar x_3 \\
        D\bar x_1 + E\bar x_3 + F\bar V
    \end{bmatrix}
\end{equation}
$\theta^e$ and $\phi^e$ have been omitted as they are equal to 0 at their equilibrium points.
\subsection{Determining Equilibrium Points} To determine equilibrium points, we can set $f(\bold x, V)=0$:
\begin{equation}
\label{sec:eq27}
    f(\bold{x^e}, V^e) = \begin{bmatrix} 
        \dot x_1^e \\ 
        \dot x_2^e \\ 
        \dot x_3^e 
    \end{bmatrix} = \begin{bmatrix}
        x_2 \\
        \frac{5}{7m}(c\frac{{x_3^e}^2}{(\delta-x_1^e)^2}+mg\sin{\phi}-k(x_1^e-d)-bx_2^e) \\
        \frac{V-x_3^eR}{L_0+L_1\exp{-\alpha(-\delta-x_1^e)}}
    \end{bmatrix} = \begin{bmatrix}
        0 \\
        0 \\
        0
    \end{bmatrix}
\end{equation}
This then defines a set of state variables $\bold{x^e}$ and an input voltage $V^e$ that satisfies equation (\ref{sec:eq27}). In the problem statement, we were told that the aim of the controller is to maintain a position of 0.5m for the ball, we can take this value and use it as our equilibrium point for $x_1$, hence $x_1^e=0.5m$. This leaves us with 3 equations and 3 unknowns, which we solved using the Python module sympy to give us the following values for $\bold{x^e}$ and $V^e$:
\begin{equation}
    \bold{x^e} = \begin{bmatrix}
        x_1^e \\
        x_2^e \\
        x_3^e
    \end{bmatrix} = \begin{bmatrix}
        0.5m \\
        0ms^-1 \\
        0.707A
    \end{bmatrix}
\end{equation}
\begin{align}
    V^e = 1556.47V
\end{align}
\section{Controller Design}
\subsection{Defining a Transfer Function} We first applied the Laplace transform to the linear approximation of the system we derived in (\ref{sec:eq26}):
\begin{align*}
    \lap\{\dot x_1(t)\} &= s\bar X_1(s) = X_2(s) \\
    \lap\{\dot x_2(t)\} &= s\bar X_2(s) = A\bar X_1(s)+B\bar X_2(s)+C\bar X_3(s) \\
    \lap\{\dot x_3(t)\} &= s\bar X_3(s) = D\bar X_1(s)+E\bar X_2(s)+F\bar V(s) \\
\end{align*}
The transfer function should relate the input voltage $\bar V$, to the position of the ball $\bar x_1$, therefore:
\begin{equation}
    G_x(s)=\frac{\bar X_1(s)}{\bar V(s)}
\end{equation}
The derivation for the transfer function can be seen below:
\begin{align}
\label{sec:eq32}
    \notag s\bar X_3(s) &= E\bar X_3(s)+F\bar V(s) \\
    \notag F\bar V(s) &= s\bar X_3(s)-E\bar X_3(s) \\
    \notag F\bar V(s) &= (s-E)\bar X_3(s) \\
    \bar X_3(s)&=\frac{F\bar V(s)}{s-E}
\end{align}
\begin{align}
\label{sec:eq33}
    \notag s\bar X_2(s) &= A\bar X_1(s)+B\bar X_2(s)+C\bar X_3(s) \\
    \notag s(s\bar X_1(s)) &= A\bar X_1(s)+B(s\bar X_1(s))+C\bar X_3(s) \\
    \notag s^2\bar X_1(s) &= A\bar X_1(s)+sB\bar X_1(s)+C\bar X_3(s) \\
    \notag C\bar X_3(s) &= s^2\bar X_1(s)-A\bar X_1(s)-sB\bar X_1(s) \\
    C\bar X_3(s) &= \bar X_1(s)(s^2-A-sB)
\end{align}
Substituting equation (\ref{sec:eq32}) into equation (\ref{sec:eq33}) gives:
\begin{align}
    \notag C(\frac{F\bar V(s)}{s-E}) &= \bar X_1(s)(s^2-A-sB) \\
    \notag \bar V(s)(\frac{CF}{s-E}) &= \bar X_1(s)(s^2-A-sB) \\
    \notag \frac{\bar X_1(s)}{\bar V(s)} &= \frac{\frac{CF}{s-E}}{s^2-A-sB} \\
    \frac{\bar X_1(s)}{\bar V(s)} &= \frac{CF}{s^3-s^2(B+E)-s(A-BE)+AE}
\end{align}
This transfer function is not BIBO stable (elaborate on this). \hyperref[fig:no_pid_imp]{Figure 2} and \hyperref[fig:no_pid_step]{Figure 3} shows the response of this transfer function to an impulse and step response respectively.
\subsection{Sensor Design}In the problem statement, we were told that the position of the ball would be measured with a sensor which could be modelled as a first-order system, the derivation of the transfer function for the sensor can be seen below: 
\begin{equation}
\label{sec:eq35}
    x(t)+\tau_m\dot x(t)=K_mu(t)
\end{equation}
To get a transfer function for the sensor, we first apply the Laplace transform to (\ref{sec:eq35}):
\begin{align*}
    X(s)+\tau_msX(s)=K_mU(s) \\
    X(s)(1+\tau_ms)=K_mU(s)
\end{align*}
The transfer function for the sensor $G_s(s)$ can then be defined as:
\begin{align}
    \frac{X(s)}{U(s)}=\frac{K_m}{1+\tau_ms}
\end{align}
In the problem statement, we were told that the time constant $\tau_m$ is equal to 30ms. The value of the static gain $K_m$ was chosen to be $1$, this was done to ensure that the output of the sensor accurately reflects the position of the ball. \\ \\
The sensor acts as a 30ms delay between the output of the system and the PID controller to emulate the effects a sensor would have on a practical system with the same design. \hyperref[fig:sensorcomparison]{Figure 4} shows the response of the PID controlled system with and without the sensor in place to display the effect on the system's response to a unit impulse.
\begin{figure}[h]
\centering
\includegraphics[width=0.6\linewidth]{pid-sensor-comparison.eps}
\caption{Response of PID controlled system with and without the sensor.}
\label{fig:sensorcomparison}
\end{figure}
\subsection{Designing a PID controller} In order to design an appropriate PID controller for this system, we defined the following constraints our controller should meet. (DEFINE THESE).
\section{Disturbance Rejection}
\section{Stability Analysis} 

\end{document}
